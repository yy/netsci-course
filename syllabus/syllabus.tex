\documentclass[11pt,article,oneside]{memoir} %{{{
% based on Kieran Healy's syllabus templates
% https://github.com/kjhealy/latex-custom-kjh

\usepackage{org-preamble-pdflatex}

\setlength{\parskip}{10pt}
\setlength{\parindent}{0pt}

% Definitions
\def\myauthor{Author}
\def\mytitle{Title}
\def\mycopyright{\myauthor}
\def\mykeywords{}
\def\mybibliostyle{plain}
\def\mybibliocommand{}
\def\mysubtitle{}
\def\myaffiliation{Indiana University}
\def\myaddress{Luddy AI Center Rm 2026}
\def\myemail{yyahn@iu.edu}
\def\myweb{http://yongyeol.com}
\def\myphone{856-2920}
\def\myversion{}
\def\myrevision{}

\def\myaffiliation{Indiana University}
\def\myauthor{Yong-Yeol ``YY'' Ahn}
\def\mykeywords{networks, network science, data science}
\def\mysubtitle{Syllabus}
\def\mytitle{{\normalsize \textsc{Info} I606 (Spring 2025) \newline} \HUGE Network Science}

%}}}
\begin{document} %{{{

%{{{ chapter style
%%\chapterstyle{article-3}
%\pagestyle{kjh}

\def\ind{\hangindent=1 true cm\hangafter=1 \noindent}
\def\labelitemi{$\cdot$}

\chapterstyle{article-4}  % alternative styles are defined in latex-custom-kjh/needs-memoir/

%}}}
%{{{ my info 
\title{\LARGE \mytitle}
\author{\Large\myauthor \newline \footnotesize\texttt{\noindent\myemail}}
%\date{\textbf{Time}: TR 9:45am--11am and Online%
\date{\textbf{Time}: Asynchronous%
  \newline \textbf{Location:} Online
\newline \textbf{Office hours}: See Canvas homepage for details.}

%\published{\sffamily I590/H400/I400 / Fall 2014 / Mon \& Wed 4:00--5:15pm / Info West 107 (M) \& 109 (W)}
\maketitle
%}}}
%{{{ TA
\vspace{-20pt}
{\bfseries Assistant Instructor} \\ 
Larry Zhang (\texttt{larzhang@iu.edu}; \texttt{@Larry Zhang})\\
%}}}

\section{Course Description} %{{{

Network science is a unifying framework to study complex systems, such as living organisms, societies, and many techno-social systems. 
Therefore, understanding networks and network (graph) data is fundamental to numerous domains. 
This graduate-level course introduces the fundamental concepts as well as key applications of network science for a broad range of students.
Topics include statistical properties and models of real-world networks, network data \& algorithms, how information and diseases spread in our society, and machine learning with networks---e.g., community detection (clustering) and graph embedding.

Because \emph{your} learning should be the primary focus, please engage actively by voicing your confusions, challenges, and intriguing digressions! Rather than merely watching lectures, think critically, engage in debates, and immerse yourself. 

%}}}
\section{Course Objectives} %{{{

By the end of the course, students are expected to be able to identify, construct, model, and analyze networks by choosing and applying appropriate methods and algorithms, as well as understanding ethical issues surrounding network data.
Students are also expected to be able to explain, both mathematically and conceptually, the key network concepts, algorithms, models, and statistical properties, as well as their implications.
%}}}
\section{Communication} %{{{

We will use Canvas and Slack for communication. \textbf{Canvas} is for official announcements as well as for anything that contains private and sensitive information. \textbf{Slack} is a more casual space for everyone taking the course this semester. Although it is not required to use it, I'd highly recommend to use it. You can use it for sharing cool resources, Q\&As, team discussions, and other casual conversations. AIs and I will be on Slack, so feel free to reach out anytime. 

%Announcements, Q\&As, and other communication will be sent via Canvas and Slack. 
%Although the most critical announcements will be sent via both platforms, a lot of course-related information (as well as questions and answers) will be shared on Slack and thus you will miss a lot of useful---although not \emph{essential} in completing the course---course-related information if you are not on Slack. 
When joining the course Slack, feel free to avoid using your full name (e.g., use ``John D.'' instead of ``John Doe'') to protect your privacy. 
Also never post your personal information or sensitive data (e.g., grades) anywhere outside Canvas or other approved IU services. 
The Slack site is:

\url{https://iu-netsci-course.slack.com/} (sign up: \url{https://iu-netsci-course.slack.com/signup})

You can create an account by using one of the following IU email addresses: \texttt{iu.edu}, \texttt{indiana.edu}, \texttt{umail.iu.edu}, \texttt{iupui.edu}.
I can also invite you with any email address. 
If you have any issues joining Slack, please let me know.

In terms of getting a response, email (and thus Canvas) will tend to be slower than Slack because I'm under a constant bombardment of emails. Slack messages can bypass that. However, \emph{if your communication contains any sensitive information}, you should use the official channels (emails or canvas messages), while you can still use slack to \emph{notify me} about your email or canvas message. 

Whenever you are not happy about the course or have a suggestion for improving the course, please share your thoughts! You can simply send a message to me, or anonymously share your opinion:

\url{https://forms.gle/9XrnrBn9faARDAsm7}

%}}}
\section{Prerequisites} %{{{ \label{sec:Prerequisites}

Although there is no formal prerequisite, the course will assume that you already have basic working knowledge of mathematics (e.g., linear algebra and calculus), probability \& statistics, programming (Python), algorithms, and data structure. 
If you do not have or are rusty with any of these foundations, there will be some reading materials that review necessary mathematical and computing concepts, as well as a notebook for reviewing basic Python programming. 
If you still do not feel ready, please talk to me or AIs so that we can provide resources that can help you learn and get prepared or catch up along the way. 

\paragraph{A note for those without strong technical backgrounds} Some course materials and assignments may present challenges if you do not have a strong technical background.
That does \emph{not} mean that you are not good enough or don't belong; it is completely natural and ok to feel this way! For context, despite majoring in Physics, my math proficiency was not very good and I struggled throughout. Much of what I teach in this class did not click for me for a while! The key is to persistently revisit the fundamentals and  put extra effort!

%}}}
\section{Requirements and Evaluation} %{{{ \label{sec:requirements}

For every module, there will be quizzes and assignments. The quizzes will test your understanding of the previous module or reading materials. If you fully understand the materials without developing substantial misconceptions, then the quizzes should be fairly straightfoward. If you are struggling with the quizzes, you may need to put more efforts at understanding the materials more deeply and may want to seek help from me, AIs, and your fellow classmates. The quizzes also let me identify common misconceptions and update the course materials and activities. 

Most assignments will ask whether you can translate your understanding of the materials into a working code. Some assignments may be challenging if you do not have lots of programming experience or computing background. Please make use of office hours as much as you can and also practice programming skills. 

The primary evaluation will be based on an exam and a class project.
The exam will consist of online and written portions. The online portion will consists of true/false, multiple choices, and fill-in-the-blank types of questions.  
Because it is difficult to see your understanding of the materials via these types of questions, there will be a written portion of the exam where you submit a scanned or typed answers with intermediate steps, which allows a more detailed assessment of your understanding.  

For your project, you'll be asked to produce a (or a series of) videos or shorts that explain cool network science. Your video can communicate your own research, dig into an interesting network data, explain a network science concept in creative ways, or anything that communicates intriguing topics of network science. I'd strongly encourage to discuss your ideas with me throughout the semester. 
The project can be conducted individually or by forming a small team.
For more information about the projects, please visit \url{https://github.com/yy/netsci-course/wiki/Projects}.

%}}}
\section{Books and key materials} %{{{

We will primarily use \emph{Working with Network Data (WWND)} by James Bagrow and yours truly:
\begin{itemize}\vspace{-10pt}
  \item \url{https://cambridgeuniversitypress.github.io/WorkingWithNetworkData/}
\end{itemize}

I will also use or refer some chapters from the following books (they are all great):
\begin{itemize}\vspace{-10pt}
  \item \emph{Network Science} by Albert-László Barabási (Cambridge University Press, 2016). \url{http://networksciencebook.com}
  \item \emph{Networks} (2nd ed.) by Mark Newman (Oxford University Press).
  \item \emph{Networks, Crowds, and Markets} by David Easley and Jon Kleinberg (Cambridge University Press, 2010).
\end{itemize}

%}}}
\section{Grading}\label{sec:grading_tentative} %{{{

If you focus on the mastery of the science, you will be able to earn good grades. The other way around may not work as well! So please focus on the mastery and fun, rather than focusing too much on grades! 

How well you are doing with quizzes will give you a good sense of how well you are understanding each module. I hope the feedbacks from quizzes and assignments will help you develop a good sense and knowledge about where you are, so that you have plenty of opportunities to keep improving your understanding of network science throughout the course. 

Note that there may be some adjustments at the end of the class due to the introduction of new materials that have not been tested before. Also, note that some assignment grades may be related late by mistake. Therefore, the grade (percentage) that you see on the Canvas may not accurately reflect your final grade.

\vspace{-10pt}
\begin{itemize}\itemsep=0pt

\item Participation (attendance, quiz, and discussion)\footnote{You may receive extra participation credits for outstanding engagement (e.g., sharing useful information or helping peers).}: 20\%  (30\% for online section)

\item Assignments: 20\%

\item Exam: 30\% (20\% for online section)

\item Project: 30\%

\end{itemize}
%}}}
\section{Course Schedule} %{{{

The schedule may change due to unexpected circumstances. See also \href{https://registrar.indiana.edu/official-calendar/index.shtml}{IU Official Calendar} for holidays, breaks, etc. 

\subsection{Key dates}\label{sub:key_dates} %{{{

Mark your calendar and plan ahead!

\vspace{-10pt}
\begin{itemize}\itemsep=0pt%{{{   
\item Project proposal due: \textbf{2/21}
%\item Project presentation (residential): \textbf{4/25} and \textbf{4/27}
\item Project team self-evaluation: \textbf{4/11}
\item Project final paper due: \textbf{5/2}
\item Final Exam: \textbf{5/4--5/8}
\end{itemize} %}}}

%}}}

\subsection{Schedule}\label{sub:schedule}%{{{

\begin{tabular}{@{}clll@{}} \toprule
  Week & Date & Topic & Project timeline\\\midrule
  01 & 1/13-- & M01: Why do we care about networks? & \\
  02 & 1/20-- & M02: Friendship paradox: a life lesson & \\
  03 & 1/27-- & M03: ``What a small world!'' \\
  04 & 2/03-- & M04: Strength of weak ties \\
  05 & 2/10-- & M05: Scale-free networks \\
  06 & 2/17-- & Project week & Proposal: 2/21 \\
  07 & 2/24-- & M06: Centralities & Project pitch: 2/28 \\
  08 & 3/03-- & M07: Communities I \\
  09 & 3/10-- & M08: Communities II \\
  10 & 3/17-- & \textbf{Spring break} \\
  11 & 3/24-- & M09: Random graphs \\
  12 & 3/31-- & M10: Network epidemiology \\
  13 & 4/07-- & M11: Social influence & Self-eval: 4/11 \\
  14 & 4/14-- & M12: Information diffusion \\
  15 & 4/21-- & M13: Machine learning with graphs  \\
  16 & 4/28-- & Project week & Presentation \& paper: 5/2 \\
  17 & 5/05-- & Final exam week & Exam: 5/4--5/8 \\ 
  \bottomrule
\end{tabular}

%}}}

%\vspace{-0.2em}\begin{itemize}\itemsep0em
%\item Network Science Ch. 2
%\item S.L. Feld, ``Why Your Friends Have More Friends Than You Do'', American Journal of Sociology 96, 1464 (1991).
%\item Steven Strogatz, \href{https://opinionator.blogs.nytimes.com/2012/09/17/friends-you-can-count-on/}{``Friends You Can Count On''}, New York Times, Sep. 17th, 2012.
%\item N.O. Hodas et al., ``Friendship Paradox Redux: Your Friends Are More Interesting Than You'', ICWSM'13
%\item Y.-H. Eom and H.-H. Jo, ``Generalized friendship paradox in complex networks: The case of scientific collaboration'', Scientific Reports 4, 4603 (2014)
%\item J.P. Bagrow et al., ``Which friends are more popular than you? Contact strength and the friendship paradox in social networks'', ASONAM'17
%\item J. Bollen et al., ``The happiness paradox: your friends are happier than you'', EPJ Data Science 6, 4 (2017).
%\end{itemize}
%}}}
%\vspace{-0.2em}\begin{itemize}\itemsep0em
%\item What is the ``Kevin Bacon game''?
%\item Who has the ``Erdős-Bacon-Sabbath number''?
%\item Stanley Milgram, ``The Small-World Problem'', Psychology Today 1, 61 (1967).
%\item Peter Sheridan Dodds et al., ``An Experimental Study of Search in Global Social Networks'', Science 301, 827 (2003).
%\item Johan Ugander et al., ``The Anatomy of the Facebook Social Graph'', arXiv:1111.4503 (2011).
%\end{itemize}
%}}}



%}}}
\section{Policies and advice} %{{{

\begin{enumerate}

% \item \emph{Let's keep everyone safe together.} 
% Let's not be a jerk to each other. 
% Masks are not required, but welcome. 
% If you are not feeling well, please stay home and rest. 
% There will be no penalty for missing classes due to COVID-19 or other health-related reasons \emph{as long as you have informed the absence before the class}. 

% \item \emph{Missing classes.} 
% If you were to miss a class, you need to notify the instructor and TAs \emph{before} the class begins to get an accommodation, except in extreme circumstances. 
% The missing attendance will be treated as ``excused'' and will not be counted towards the final grade.
% You can use the videos and slides (and office hours) to catch up with the class. 

\item \emph{Be honest.} Your assignments and papers should be your own work.  
If you find useful resources for your assignments, share them and cite them. 
If your friends helped you, acknowledge them. 
Feel free to discuss both online and offline, but you should not show your solution nor see others'.  
Any cases of serious academic misconduct (cheating, fabrication, plagiarism, etc) will be reported to the School and the Dean of Students, following the standard procedure. 
But more than anything, cheating will hurt you in the long term and \emph{not cool}.

\item \emph{You have the responsibility of backing up all your data and code}.
Always back up your work with a cloud storage service such as Google Drive or One Drive. 
Learn version control systems and use \url{https://github.iu.edu} or \url{https://github.com}. 
The loss of data, code or papers due to the lack of backup is not an acceptable excuse.

\item \emph{Late assignments}. There will be a 10\% late penalty for the late assignments unless excused before the deadline. 

\item \emph{Disabilities.} Every attempt will be made to accommodate qualified students with disabilities (e.g., mental health, learning, chronic health, physical, hearing, vision, neurological, etc.). 
You must have established your eligibility for support services through Disability Services for Students. 
Note that services are confidential, may take time to put into place, and are not retroactive.  
Captions and alternate media for print materials may take three or more weeks to get produced. 
Please contact Disability Services for Students at \url{http://disabilityservices.indiana.edu} or 812-855-7578 as soon as possible if accommodations are needed. 
The office is located on the third floor, west tower, of the Wells Library (Room W302). 
Walk-ins are welcome 8 AM to 5 PM, Monday through Friday. 
You can also locate a variety of campus resources for students and visitors who need assistance at \url{http://www.iu.edu/~ada/index.shtml}.

\item \emph{Bias-based incidents.} Any act of discrimination or harassment based on 
race, ethnicity, religious affiliation, gender, gender identity, sexual orientation, or
disability can be reported to \texttt{biasincident@indiana.edu} or to the Dean of Students Office at (812) 855-8188. 

\item \emph{Sexual misconduct and Title IX.} As your instructor, one of my
responsibilities is to create a positive learning environment for all students.
Title IX and IU's Sexual Misconduct Policy prohibit sexual misconduct in any
form, including sexual harassment, sexual assault, stalking, and dating, and
domestic violence.  If you have experienced sexual misconduct, or know someone
who has, the University can help. If you are seeking help and would like to
speak to someone confidentially, you can make an appointment with:

\begin{enumerate}

\item The Sexual Assault Crisis Services (SACS) at (812) 855-8900 (counseling services)
\item Confidential Victim Advocates (CVA) at (812) 856-2469 (advocacy and advice services)
\item IU Health Center at (812) 855-4011 (health and medical services)

\end{enumerate}

It is also important that you know that Title IX and University policy require
me to share any information brought to my attention about potential sexual
misconduct, with the campus Deputy Title IX Coordinator or IU's Title IX
Coordinator. In that event, those individuals will work to ensure that
appropriate measures are taken and resources are made available. Protecting
student privacy is of utmost concern, and information will only be shared with
those that need to know to ensure the University can respond and assist. I
encourage you to visit \emph{stopsexualviolence.iu.edu} to learn more.

\item \emph{Professionalism and personal misconduct.} In this course, students are expected to communicate respectfully with peers, instructors, and teaching assistants. This includes using appropriate language in emails, discussions, and assignments, as well as being mindful of the tone and timing of requests or feedback. Unprofessional behavior and misconduct may result in penalties or be reported to the Office of Student Conduct.  

\end{enumerate}

%}}}
\end{document} %}}}
