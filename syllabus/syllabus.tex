\documentclass[11pt,article,oneside]{memoir} %{{{
% based on Kieran Healy's syllabus templates
% https://github.com/kjhealy/latex-custom-kjh

\usepackage{org-preamble-pdflatex}

\setlength{\parskip}{10pt}
\setlength{\parindent}{0pt}

% Definitions
\def\myauthor{Author}
\def\mytitle{Title}
\def\mycopyright{\myauthor}
\def\mykeywords{}
\def\mybibliostyle{plain}
\def\mybibliocommand{}
\def\mysubtitle{}
\def\myaffiliation{Indiana University}
\def\myaddress{Luddy AI Center Rm 2026}
\def\myemail{yyahn@iu.edu}
\def\myweb{http://yongyeol.com}
\def\myphone{856-2920}
\def\myversion{}
\def\myrevision{}

\def\myaffiliation{Indiana University}
\def\myauthor{Yong-Yeol ``YY'' Ahn}
\def\mykeywords{networks, network science, data science}
\def\mysubtitle{Syllabus}
\def\mytitle{{\normalsize \textsc{Info} I606 (Spring 2023) \newline} \HUGE Network Science}

%}}}
\begin{document} %{{{

%{{{ chapter style
%%\chapterstyle{article-3}
%\pagestyle{kjh}

\def\ind{\hangindent=1 true cm\hangafter=1 \noindent}
\def\labelitemi{$\cdot$}

\chapterstyle{article-4}  % alternative styles are defined in latex-custom-kjh/needs-memoir/

%}}}
%{{{ my info 
\title{\LARGE \mytitle}
\author{\Large\myauthor \newline \footnotesize\texttt{\noindent\myemail}}
\date{\textbf{Time}: TR 9:45am--11am and Online%
  \newline \textbf{Location:} LH 101 
\newline \textbf{Office hours}: Monday 12pm-1pm ET on Zoom. \\You can also stop by my office (LU2026) anytime; also anytime on Slack.}

%\published{\sffamily I590/H400/I400 / Fall 2014 / Mon \& Wed 4:00--5:15pm / Info West 107 (M) \& 109 (W)}
\maketitle
%}}}
%{{{ TA
\vspace{-20pt}
{\bfseries Assistant Instructor} \\ 
Vincent Wong (\texttt{vmwong@iu.edu})\\
Rachith Aiyappa (\texttt{racball@iu.edu})\\
Ashutosh Tiwari (\texttt{ashutiwa@iu.edu})\\
Ashutosh Hathidara (\texttt{ashuhath@iu.edu})
%)Elise Jing (\texttt{jingy@iu.edu}); Officours: Monday 2pm-3pm \\ Online: Jaehyuk Park (\texttt{jp70@iu.edu}); Office Hours: Friday 2pm-3pm}}
%}}}

\section{Course Description} %{{{

Network science is a unifying framework to study complex systems, such as living organisms, societies, and many techno-social systems. 
Therefore, understanding networks and network (graph) data is fundamental to numerous domains. 
This graduate-level course introduces the fundamental concepts as well as key applications of network science.
Topics include statistical properties and models of real-world networks, network data \& algorithms, how information and diseases spread in our society, and machine learning with networks---e.g., community detection (clustering) and graph embedding.

%}}}
\section{Course Objectives} %{{{

By the end of the course, students are expected to be able to identify, construct, model, and analyze networks by choosing and applying appropriate methods and algorithms, as well as understanding ethical issues surrounding network data.
Students are also expected to be able to explain, both mathematically and conceptually, the key network concepts, algorithms, models, and statistical properties, as well as their implications.
%}}}
\section{Communication} %{{{

We will use Canvas and Slack for communication. \textbf{Canvas} is for official announcements as well as for anything that contains private and sensitive information. \textbf{Slack} is for day-to-day information sharing, Q\&As, team discussions, and other casual conversations. 

Announcements, Q\&As, and other communication will be sent via Canvas and Slack. 
Although the most critical announcements will be sent via both platforms, a lot of course-related information (as well as questions and answers) will be shared on Slack and thus you will miss a lot of useful---although not \emph{essential} in completing the course---course-related information if you are not on Slack. 
When joining the course Slack, feel free to avoid using your full name (e.g., use ``John D.'' instead of ``John Doe'') to protect your privacy. 
Also never post your personal information or sensitive data (e.g., grades) anywhere outside Canvas or other approved IU services. 
The Slack site is:

\url{https://iu-netsci-course.slack.com/}

You can join by visiting the signup URL:

\url{https://iu-netsci-course.slack.com/signup}

You can create an account by using one of the following IU email addresses: \texttt{iu.edu}, \texttt{indiana.edu}, \texttt{umail.iu.edu}, \texttt{iupui.edu}.
I can also invite you with any email address. 
If you have any issues joining Slack, please contact the instructor.

Email (and thus Canvas) will be much slower than Slack because the instructors are under a constant bombardment of emails. \emph{If your communication contains sensitive information}, you should use the official channels (emails or canvas messages), but you can still use slack to \emph{notify me} about your email or canvas message. Please expect the following response time:

\begin{itemize}
\item \textbf{Email or Canvas}: the instructors will probably be able to respond within \emph{one week} (likely within several days).
\item \textbf{Slack}: the instructors will probably be able to respond within \emph{one day} (likely within several hours).

\end{itemize}

Whenever you are not happy about the course or have a suggestion for improving the course, please share your thoughts! You can simply send a message on slack, or anonymously share your opinion:

\url{https://forms.gle/9XrnrBn9faARDAsm7}

%}}}
\section{Prerequisites} %{{{ \label{sec:Prerequisites}

Although there is no formal prerequisite, the course will require a good foundation of mathematics, statistics, programming, algorithms, and data structure. 
Python is used as the main programming language. 
Please contact the instructor if you are uncertain about your background.

%}}}
\section{Requirements and Evaluation} %{{{ \label{sec:requirements}

Students are expected to read reading assignments, attend class meetings, and complete quizzes and assignments. 
This course is not driven by one-way lectures, but by your participation and engagement.
So be prepared to think, calculate with a pen \& paper, and debate vigorously with classmates. 

The main evaluation will be based on an exam and a class project.
The project can be conducted individually or by forming a small team.
Students may choose any network-related topics that involve network analysis or modeling, although it is strongly encouraged to seek guidance from the instructor.
For more information about the projects, please visit \url{https://github.com/yy/netsci-course/wiki/Projects}.

%}}}
\section{Books and key materials} %{{{

We will sample from multiple sources. 
\href{http://barabasi.com/networksciencebook/}{Network Science} by Albert-László
Barabási and
\href{https://www.amazon.com/Networks-Mark-Newman/dp/0198805098}{Networks:
An Introduction} by Mark Newman, as well as some materials from the instructor. See also \url{https://yyiki.org/wiki/Network%20science/} 

%}}}
\section{Grading}\label{sec:grading_tentative_} %{{{

Note that there may be some adjustments (e.g., a curve) at the end of the class. Also, some grading may happen late and some grades may not be released by mistake. Therefore, the grade (percentage) that you see on the Canvas may not necessarily reflect your final grade.

\vspace{-10pt}
\begin{itemize}\itemsep=0pt

\item Participation (attendance, quiz, and discussion)\footnote{There may be extra participation credits for sharing useful information and helping others.}: 20\%  (30\% for online section)

\item Assignments: 20\%

\item Exam: 30\% (20\% for online section)

\item Project: 30\%

\end{itemize}
%}}}
\section{Course Schedule} %{{{

The schedule may change due to unexpected circumstances. See also \href{https://registrar.indiana.edu/official-calendar/index.shtml}{IU Official Calendar} for holidays, breaks, etc. 

\subsection{Key dates}\label{sub:key_dates} %{{{

Mark your calendar and plan ahead!

\vspace{-10pt}
\begin{itemize}\itemsep=0pt%{{{
\item Project proposal due: \textbf{3/3}
\item Project presentation (residential): \textbf{4/25} and \textbf{4/27}
\item Project final paper due: \textbf{5/5}
\item Final Exam: TBD, During the final week of the semester (5/1-5/5)
\end{itemize} %}}}

%}}}

\subsection{Schedule}\label{sub:schedule}%{{{

\begin{tabular}{@{}clll@{}} \toprule
  Week & Date & Topic & Project timeline\\\midrule
  01 & 1/08-- & M01: Why should we care about networks? & \\
  02 & 1/15-- & M02: Friendship paradox: a life lesson & \\
  03 & 1/22-- & M03: ``What a small world!'' \\
  04 & 1/29-- & M04: Strength of weak ties \\
  05 & 2/05-- & M05: Scale-free networks \\
  06 & 2/12-- & M06: Random graphs & Proposal: 2/16 \\
  07 & 2/19-- & M07: Centralities \\
  08 & 2/26-- & M08: Communities and other structures \\
  09 & 3/04-- & M09: Structural inference \\
  10 & 3/11-- & \textbf{Spring break} \\
  11 & 3/18-- & M10: Network epidemiology & Check-in: 3/22\\
  12 & 3/25-- & M11: Information diffusion \\
  13 & 4/01-- & M12: Graph embedding\\
  14 & 4/08-- & \textbf{Exam} \\
  15 & 4/15-- & Project hackweek \\
  16 & 4/22-- & Project Presentation \\
  17 & 4/29-- & Final exam week & Project Due: 5/3 \\ 
  \bottomrule
\end{tabular}

%}}}

%\vspace{-0.2em}\begin{itemize}\itemsep0em
%\item Network Science Ch. 2
%\item S.L. Feld, ``Why Your Friends Have More Friends Than You Do'', American Journal of Sociology 96, 1464 (1991).
%\item Steven Strogatz, \href{https://opinionator.blogs.nytimes.com/2012/09/17/friends-you-can-count-on/}{``Friends You Can Count On''}, New York Times, Sep. 17th, 2012.
%\item N.O. Hodas et al., ``Friendship Paradox Redux: Your Friends Are More Interesting Than You'', ICWSM'13
%\item Y.-H. Eom and H.-H. Jo, ``Generalized friendship paradox in complex networks: The case of scientific collaboration'', Scientific Reports 4, 4603 (2014)
%\item J.P. Bagrow et al., ``Which friends are more popular than you? Contact strength and the friendship paradox in social networks'', ASONAM'17
%\item J. Bollen et al., ``The happiness paradox: your friends are happier than you'', EPJ Data Science 6, 4 (2017).
%\end{itemize}
%}}}
%\vspace{-0.2em}\begin{itemize}\itemsep0em
%\item What is the ``Kevin Bacon game''?
%\item Who has the ``Erdős-Bacon-Sabbath number''?
%\item Stanley Milgram, ``The Small-World Problem'', Psychology Today 1, 61 (1967).
%\item Peter Sheridan Dodds et al., ``An Experimental Study of Search in Global Social Networks'', Science 301, 827 (2003).
%\item Johan Ugander et al., ``The Anatomy of the Facebook Social Graph'', arXiv:1111.4503 (2011).
%\end{itemize}
%}}}



%}}}
\section{Policies and advice} %{{{

\begin{enumerate}

\item \emph{Let's keep everyone safe together.} 
Let's not be a jerk to each other. 
Masks are not required, but welcome. 
If you are not feeling well, please stay home and rest. 
There will be no penalty for missing classes due to COVID-19 or other health-related reasons \emph{as long as you have informed the absence before the class}. 

\item \emph{Missing classes.} 
If you were to miss a class, you need to notify the instructor and TAs \emph{before} the class begins to get an accommodation, except in extreme circumstances. 
The missing attendance will be treated as ``excused'' and will not be counted towards the final grade.
You can use the videos and slides (and office hours) to catch up with the class. 

\item \emph{Be honest.} Your assignments and papers should be your own work.  
If you find useful resources for your assignments, share them and cite them. 
If your friends helped you, acknowledge them. 
Feel free to discuss both online and offline, but you should not show your solution nor see others'.  
Any cases of serious academic misconduct (cheating, fabrication, plagiarism, etc) will be reported to the School and the Dean of Students, following the standard procedure. 
But more than anything, cheating will hurt you in the long term and \emph{not cool}.

\item \emph{You have the responsibility of backing up all your data and code}.
Always use at least a cloud storage services such as Google Drive or One Drive. 
Learn version control systems and use \url{https://github.iu.edu} or \url{https://github.com}. 
The loss of data, code or papers due to the lack of any backup is not an acceptable excuse.

%\item \emph{Late assignments}. There will be a 10\% late penalty for the late assignments unless excused. 

\item \emph{Disabilities.} Every attempt will be made to accommodate qualified students with disabilities (e.g., mental health, learning, chronic health, physical, hearing, vision, neurological, etc.). 
You must have established your eligibility for support services through Disability Services for Students. 
Note that services are confidential, may take time to put into place, and are not retroactive.  
Captions and alternate media for print materials may take three or more weeks to get produced. 
Please contact Disability Services for Students at \url{http://disabilityservices.indiana.edu} or 812-855-7578 as soon as possible if accommodations are needed. 
The office is located on the third floor, west tower, of the Wells Library (Room W302). 
Walk-ins are welcome 8 AM to 5 PM, Monday through Friday. 
You can also locate a variety of campus resources for students and visitors who need assistance at \url{http://www.iu.edu/~ada/index.shtml}.

\item \emph{Bias-based incidents.} Any act of discrimination or harassment based on 
race, ethnicity, religious affiliation, gender, gender identity, sexual orientation, or
disability can be reported to \texttt{biasincident@indiana.edu} or to the Dean of Students Office at (812) 855-8188. 

\item \emph{Sexual misconduct and Title IX.} As your instructor, one of my
responsibilities is to create a positive learning environment for all students.
Title IX and IU's Sexual Misconduct Policy prohibit sexual misconduct in any
form, including sexual harassment, sexual assault, stalking, and dating, and
domestic violence.  If you have experienced sexual misconduct, or know someone
who has, the University can help. If you are seeking help and would like to
speak to someone confidentially, you can make an appointment with:

\begin{enumerate}

\item The Sexual Assault Crisis Services (SACS) at (812) 855-8900 (counseling services)
\item Confidential Victim Advocates (CVA) at (812) 856-2469 (advocacy and advice services)
\item IU Health Center at (812) 855-4011 (health and medical services)

\end{enumerate}

It is also important that you know that Title IX and University policy require
me to share any information brought to my attention about potential sexual
misconduct, with the campus Deputy Title IX Coordinator or IU's Title IX
Coordinator. In that event, those individuals will work to ensure that
appropriate measures are taken and resources are made available. Protecting
student privacy is of utmost concern, and information will only be shared with
those that need to know to ensure the University can respond and assist. I
encourage you to visit \emph{stopsexualviolence.iu.edu} to learn more.


\end{enumerate}
%}}}
\end{document} %}}}
